\documentclass[12pt]{book}
%\usepackage[cp1251]{inputenc}
\usepackage[english,russian]{babel}
\usepackage{amssymb,latexsym,amsmath,euscript,amsfonts,amsthm}\usepackage{graphicx}
\usepackage{color}

\usepackage{fontspec}
\usepackage{polyglossia}
\setdefaultlanguage{russian}
\setmainfont[Mapping=tex-text]{CMU Serif}

\title{Дифференцируемые динамические системы}
\author{Лан Вен}

\newtheorem{theorem}{Теорема}[chapter]
\newtheorem{lemma}{Лемма}[chapter]
\newtheorem*{corollary}{Следствие}
\newtheorem{definition}{Определение}[chapter]
\newtheorem*{conjecture}{Гипотеза}
\newtheorem*{example}{Пример}
\newtheorem*{remark}{Замечание}

\newcounter{exercise}
\newenvironment{exercise}{\refstepcounter{exercise}\smallskip\noindent\textbf{Упражнение~\theexercise.}}{}
\numberwithin{exercise}{chapter}

\newenvironment{demo}{\noindent\textsl{Доказательство. }}{ $\square$ \medskip}

\begin{document}
\pagestyle{plain}
\makeatletter
\@input{preface}
\@input{ch1_introduction}
\@input{ch1_1}
\@input{ch1_2}
\@input{ch1_3}
\@input{ch1_4}
\@input{ch1_exercises}
\@input{ch2_introduction}
\@input{ch2_1}
\@input{ch2_2}
\@input{ch2_3}
\@input{ch2_4}
\@input{ch2_5}
\@input{ch2_exercises}
\@input{ch3_introduction}
\@input{ch3_1}
\@input{ch3_2}
\@input{ch3_3}
\@input{ch3_4}
\@input{ch3_exercises}
\@input{ch4_introduction}
\@input{ch4_1}
\@input{ch4_2}
\@input{ch4_3}
\@input{ch4_4}
\@input{ch4_5}
\@input{ch4_6}
\@input{ch4_exercises}
\@input{ch5_introduction}
\@input{ch5_1}
\@input{ch5_2}
\@input{ch5_3}
\@input{ch5_4}
\@input{ch5_exercises}
\@input{ch6_introduction}
\@input{ch6_1}
\@input{ch6_2}
\@input{ch6_3}
\@input{ch6_4}
\@input{ch6_5}
\@input{ch6_exercises}
\@input{bibliography}
\makeatother
\end{document}
